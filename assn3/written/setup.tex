% general setup for a homework assignment

\author{15-312: Principles of Programming Languages (Fall 2023)}
\date{}

\usepackage{ifthen}
\usepackage{xparse}
\usepackage{fullpage}
\usepackage[parfill]{parskip}
\usepackage[utf8]{inputenc}
\usepackage{amsmath}
\usepackage{amssymb}
\usepackage{stmaryrd}
\usepackage{amsthm}
\usepackage{mathtools}
\usepackage{proof}
\usepackage{colonequals}
\usepackage{comment}
\usepackage{textcomp}
\usepackage[us]{optional}
\usepackage{color}
\usepackage{url}
\usepackage{verbatim}
\usepackage{graphics}
\usepackage{mathpartir}
\usepackage{enumitem}
\usepackage{tikz}

% these two are used to create the wavy division sign
\usepackage{stackengine}
\usepackage{scalerel}

\usepackage{hyperref}
\usepackage[nameinlink, capitalise]{cleveref}

\usepackage{titlesec}
\newcommand{\sectionbreak}{\clearpage}

\renewrobustcmd{\path}[1]{\colorbox{green!60!black!12}{\texttt{\detokenize{#1}}}}

\usepackage{listings}

% fix line numbering with linerange
\makeatletter
\lst@Key{matchrangestart}{f}{\lstKV@SetIf{#1}\lst@ifmatchrangestart}
\def\lst@SkipToFirst{%
    \lst@ifmatchrangestart\c@lstnumber=\numexpr-1+\lst@firstline\fi
    \ifnum \lst@lineno<\lst@firstline
        \def\lst@next{\lst@BeginDropInput\lst@Pmode
        \lst@Let{13}\lst@MSkipToFirst
        \lst@Let{10}\lst@MSkipToFirst}%
        \expandafter\lst@next
    \else
        \expandafter\lst@BOLGobble
    \fi}
\makeatother

\definecolor{background_color}{RGB}{240, 240, 240}
\definecolor{string_color}    {RGB}{180, 156,   0}
\definecolor{keyword_color}   {RGB}{ 64, 100, 255}
\definecolor{comment_color}   {RGB}{  0, 117, 110}
\definecolor{number_color}    {RGB}{ 84,  84,  84}
\lstset{
  basicstyle=\ttfamily,
  breaklines=true,
  breakatwhitespace,
  numberstyle=\tiny\ttfamily\color{number_color},
  matchrangestart=t,
  rulecolor=\color{black},
  backgroundcolor=\color{background_color},
  stringstyle=\color{string_color},
  showstringspaces=false,
  keywordstyle=\color{keyword_color},
  commentstyle=\color{comment_color},
  alsoletter={\\},
  literate=
    {`}{{\char18}}1
    {λ}{{$\lambda$}}1
    {τ}{{$\tau$}}1
    {Γ}{{$\Gamma$}}1
    {⊢}{{$\vdash$}}1
    {μ}{{$\mu$}}1
    {ρ}{{$\rho$}}1
    {ν}{{$\nu$}}1
    {σ}{{$\sigma$}}1
}

\lstdefinelanguage{sml}{
    language=ML,
    morestring=[b]",
    morecomment=[s]{(*}{*)},
    morekeywords={
        bool, char, exn, int, real, string, unit, list, option,
        EQUAL, GREATER, LESS, NONE, SOME, nil,
        andalso, orelse, true, false, not,
        if, then, else, case, of, as,
        let, in, end, local, val, rec,
        datatype, type, exception, handle,
        fun, fn, op, raise, ref,
        structure, struct, signature, sig, functor, where,
        include, open, use, infix, infixr, o, print
    }
}

% code inline
\newrobustcmd{\code}[2][]{{\sloppy
\ifmmode
    \text{\colorbox{background_color}{\lstinline[language=sml,#1]`#2`}}
\else
    {\colorbox{background_color}{\lstinline[language=sml,#1]`#2`}}%
\fi}}
% code block
\lstnewenvironment{codeblock}[1][]{\lstset{language=sml,numbers=none,#1}}{}
% code file
\newrobustcmd{\codefile}[2][]{%
  \lstinputlisting%
    [language=sml,mathescape=false,frame=single,xleftmargin=3pt,xrightmargin=3pt,title={\path{#2}},numbers=left,#1]%
    %{../../src/dist/#2}
    {../#2}
}

\usepackage{caption}
\DeclareCaptionFormat{listing}{\hfill#1#2#3\vskip1pt}
\captionsetup[lstlisting]{format=listing,singlelinecheck=false, margin=0pt, font={sf},labelsep=space,labelfont=bf}

\usepackage{chngcntr}
\counterwithin{figure}{section}


\usepackage{newunicodechar}
\newunicodechar{λ}{$\lambda$}

%\usepackage{exercises}

%%% exercise infrastructure, adapted from pfpl

\allowdisplaybreaks[1]       %mildly permissible to break up displayed equations

\theoremstyle{plain}
\newtheorem{theorem}{Theorem}[section]
\newtheorem{lemma}[theorem]{Lemma}
\newtheorem{corollary}[theorem]{Corollary}
\theoremstyle{definition}
\newtheorem*{remark}{Remark}
\newtheorem{defn}[theorem]{Definition}
\newtheorem{example}[theorem]{Example}

\NewDocumentCommand{\Infer}{s o m m}{%
  \IfBooleanTF{#1}%
    {\IfNoValueTF{#2}%
      {\inferrule{#3}{#4}}%
      {\inferrule*[right={#2}]{#3}{#4}}}%
    {\IfNoValueTF{#2}%
      {\inferrule{#3}{#4}}%
      {\inferrule*[Right={#2}]{#3}{#4}}}%
}

\newenvironment{infrules}[1]{\begin{subequations}\label{rules:#1}}{\end{subequations}\ignorespacesafterend}
\newenvironment{infrule}[1]{\begin{equation}\label{rule:#1}\vcenter\bgroup}{\egroup\end{equation}\ignorespacesafterend}

\newenvironment{synchart}[1]%
{\begin{equation*}
      \label{eqn:#1}
      \renewcommand{\bnfalt}{\mathrel{\phantom{\mid}}}
      \begin{array}{llc@{\quad\extracolsep{\fill}}lll}
}
{\end{array}\end{equation*}\ignorespacesafterend}

\newenvironment{synchart*}[1]%
{\begin{equation*}
      \label{syn:#1}
      \renewcommand{\bnfalt}{\mathrel{\phantom{\mid}}}
      \begin{array}{llc@{\quad\extracolsep{\fill}}lll}
        \textit{Sort} & & & \textit{Abstract} & \textit{Concrete} & \textit{Description} \\
}
{\end{array}\end{equation*}\ignorespacesafterend}

\newcounter{taskcounter}
\newcounter{taskPercentCounter}
\newcounter{taskcounterSection}
\setcounter{taskcounter}{1}
\setcounter{taskPercentCounter}{0}
\setcounter{taskcounterSection}{\value{section}}
\newcommand{\mayresettaskcounter}{\ifthenelse{\value{taskcounterSection} < \value{section}}
{\setcounter{taskcounterSection}{\value{section}}\setcounter{taskcounter}{1}}
{}}

%% part of a problem
\newcommand{\task}[1]
  {{\bf Task\mayresettaskcounter{}\addtocounter{taskPercentCounter}{#1} \arabic{section}.\arabic{taskcounter}\addtocounter{taskcounter}{1}} (#1 pts).}

\newcommand{\ectask}
  {\bigskip \noindent {\bf Task\mayresettaskcounter{} \arabic{section}.\arabic{taskcounter}\addtocounter{taskcounter}{1}} (Extra Credit).}

\newcommand{\problemset}[1]
  {\ifthenelse{\equal{\issolution}{true}}
  {}{{#1}}}

% Solution-only - uses an "input" so that it's still safe to publish the problem set file
\definecolor{solutioncolor}{rgb}{0.0, 0.0, 0.5}
\newcommand{\solution}[1]
  {\ifthenelse{\equal{\issolution}{true}}
  {\begin{quote}
    \addtocounter{taskcounter}{-1}
    \fbox{\textcolor{solutioncolor}{\bf Solution \arabic{section}.\arabic{taskcounter}}}
    \addtocounter{taskcounter}{1}
    \textcolor{solutioncolor}{\input{solutions/#1}}
  \end{quote}}
  {}%
  \vspace{2em}}


\newtheorem*{hint}{Hint}

\usepackage{import}

\newcommand{\PFPL}[1][]{\textbf{\textsf{PFPL}}\ifthenelse{\equal{#1}{}}{}{, Chapter~#1}}